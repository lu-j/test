Lizenz: Creative Commons CC-BY-SA

\section{Einleitung}

\subsection{Ziel dieses Dokuments}

Ziel dieses Dokuments ist es, einen Diskurs über einen offenen Standard
zum Datenabruf aus Ratsinformationssystemen in Gang zu bringen.

Ratsinformationssysteme (RIS) werden von vielen Körperschaften wie
Kommunen, Landkreisen und Regierungsbezirken eingesetzt, um die
anfallende Gremienarbeit (Ratssitzungen, Ausschüsse, Vertretungen) zu
organisieren. Da ein großer Teil der schriftlichen Arbeit der
Lokalpolitik über derartige Systeme verwaltet wird, sind die RIS -- dort
wo vorhanden -- ein wichtiger Zugriffspunkt für alle, die sich für
politischen Geschehnisse interessieren.

Eine wichtige Maßnahme von Körperschaften, die im Zuge von Open-Data-
und Open-Government-Initiativen ihre Politik transparenter machen
wollen, wird auch sein, die Daten in den RIS im Sinne des
Open-Data-Begriffs zugänglich zu machen. Hierdurch können die Kommunen
selbst, aber auch dritte, Anwendungen entwickeln, die Inhalte auf
verschiedene Art und Weise auswerten, abrufbar und nutzbar machen, sei
es für die Allgemeinheit oder für bestimmte Nutzerkreise.

In Deutschland gibt es über 10.000 Kommunen, außerdem hunderte weitere
Körperschaften, die über RIS-ähnliche Systeme verfügen. Sollten diese
beginnen, ihre RIS zu öffnen, werden sie sämtlich vor der Frage stehen,
wie Daten zu modellieren und Schnittstellen (APIs) zu spezifizieren
sind.

Sowohl die Anbieter der Daten, als auch die Abnehmer (die
Anwendungsentwickler) werden von einer Standardisierung der
Schnittstellen und Datenmodelle profitieren. So wird die Kompatibilität
von Software und die breite Einsetzbarkeit ermöglicht.

Dieses Dokument soll die Erarbeitung eines solchen Standards ermöglichen
und als Diskussionsgrundlage dienen.

\subsection{Status}

Dieser Entwurf gibt aktuell einen Vorschlag des Autors wieder. Bisher
ist noch kein Feedback eingeflossen.

\subsection{Überblick}

Der Entwurf umfasst im ersten Schritt die abstrakte Beschreibung eines
Datenmodells.

\subsection{Nächste Schritte}

\begin{itemize}
\item
  Bis Ende Januar 2012: Einsammeln von Feedback zum Entwurf des
  Datenmodells Anpassen des Entwurfs anhand von Feedback
\item
  Erarbeitung eines Entwurfs für eine REST-Schnittstelle
\end{itemize}

\subsection{Adresse für Feedback}

Feedback kann gerne per Mail an marian@sendung.de übermittelt werden.
Wer Feedback übersendet, wird als Mitwirkender in zukünftigen Versionen
des Dokuments namentlich erwähnt. Wer dies nicht möchte, sollte dies
bitte in seiner Mail bitte explizit erwähnen.

\section{Datenmodell}

Das Datenmodell soll die Bausteine für die später zu entwerfende
Schnittstelle definieren. Im folgenden werden sozusagen die Objekttypen
bzw. die Klassen beschrieben, auf die über eine spätere API zugegriffen
werden kann.

Einige Objekte werden eine eindeutige Identifizierung (ID) benötigen,
wobei „eindeutig`` auch eine Frage des Kontextes ist. In den wenigsten
Fällen wird es notwendig sein, eine Objekt-Kennung weltweit eindeutig zu
machen. Darüber hinaus wird zu entscheiden sein, ob IDs unveränderlich
oder veränderlich sein sollen.

\subsection{Gebietskörperschaft}

Die Gebietskörperschaft erlaubt es, Körperschaften wie einen bestimmten
Landkreis, eine bestimmte Gemeinde oder einen bestimmten Stadtbezirk in
Form eines Datenobjekts abzubilden.

Viele RIS werden nur genau eine Instanz dieses Typs „beherbergen``.
Denkbar ist aber auch, dass Systeme für einen Verbund von mehreren
Körperschaften betrieben werden.

\begin{figure}[htbp]
\centering
\includegraphics{images/01.png}
\caption{Abbildung: Gebietskörperschaft}
\end{figure}

\subsubsection{Eindeutige Identifizierung}

Zur Identifizierung des Objekts kann der Amtliche Gemeindeschlüssel
(AGS{[}1{]}) verwendet werden, der für Deutschland alle Gemeinden,
Landkreise, kreisfreien Städte etc. eindeutig erfasst.

Vorteil der Verwendung des AGS:

\begin{itemize}
\item
  Kompakte, einfache und einheitliche Schreibweise für jede
  Körperschaft.
\item
  Der AGS wird von Behörden genutzt, ist anerkannt und auch in anderen
  Medien, z.B. der Wikipedia, verbreitet.
\end{itemize}

Nachteil des AGS:

\begin{itemize}
\item
  Führende Nullen machen den Schlüssel fehleranfällig. Bestimmte Systeme
  wie z.B. Excel könnten den Inhalt als Zahlenwert erkennen und die
  führenden Nullen automatisch verwerfen.
\end{itemize}

\subsubsection{Attribute}

\begin{description}
\item[Name]
Der Name der Gebietskörperschaft, z.B. ``Köln'' oder ``Stadt Köln''.
\end{description}

\subsection{Gremium}

Das Gremium ist ein Personenkreis, üblicherweise von gewählten und/oder
ernannten Mitgliedern. Beispiele hierfür sind der Stadtrat, Kreisrat,
Gemeinderat, Ausschüsse und Bezirksvertretungen. Gremien halten
Sitzungen ab, zu denen die Gremien-Mitglieder eingeladen werden.

\subsection{Person}

Jede natürliche Person, die Mitglied eines Gremiums ist, ist als Person
im Datenmodell eindeutig identifizierbar.

\subsection{Organisation}

Organisationen sind üblicherweise Parteien bzw. Fraktionen, denen die
Personen angehören können.

\subsection{Sitzung}

Eine Sitzung ist die Versammlung der Mitglieder eines Gremiums zu einem
bestimmten Zeitpunkt. Sitzungen können eine laufende Nummer haben.,
üblicherweise beginnend bei 1 zu Beginn einer Wahlperiode, haben.

Die geladenen Teilnehmer der Sitzung sind jeweils als „Person`` in
entsprechender Form referenziert. Verschiedene Drucksachen (Einladung,
Ergebnis- und Wortprotokoll) werden ebenfalls referenziert.

\subsection{Tagesordnungspunkt}

Der Tagesordnungspunkt wird für eine bestimmte Sitzung angelegt, erhält
eine (innerhalb dieser Sitzung eindeutige) Nummer und einen Titel
(Betreff). Nach der Sitzung wird dem Tagesordnungspunkt außerdem ein
Ergebnis angehängt. Falls abweichend von der ursprünglichen
Beschlussvorlage (z.B. durch Berücksichtigung eines Änderungsantrags)
kann ein bestimmter Beschlusstext zu Protokoll gegeben werden. Sofern
das Abstimmungsergebnis nicht einstimmig ist, kann es durch mehrere
referenzierende Stimmabgaben festgehalten werden.

\subsection{Stimmabgabe}

Wie eine Person bzw. eine Fraktion zu einem Tagesordnungspunkt
abgestimmt hat, wird durch eine Stimmabgabe festgehalten. Ganze
Abstimmungsergebnisse bestehen überlicherweise aus mehreren
Stimmabgaben. Jede Stimmabgabe gibt entweder die (einzelne) Stimme einer
Peson wieder, in diesem Fall ist die Anzahl der Stimmen zwingend 1. Oder
eine Stimmabgabe gibt das Abstimmungsverhalten einer ganzen Gruppe von
Personen wieder. Dann ist die Anzahl der Stimmen anzugeben und statt
einer Person eine Organisation (in der Regel die Fraktion) zu
referenzieren.

\subsection{Drucksache}

Eine Drucksache bildet Mitteilungen, Antworten auf Anfragen,
Beschlussvorlagen, Anfragen und Anträge ab. Jede Drucksache erhält eine
eindeutige Kennung. Das Datum gibt an, wann die Drucksache erzeugt bzw.
veröffentlicht wurde.

Die Drucksache verweist auf genau ein Hauptdokument. Darüber hinaus
können beliebig viele Dokumente als Anhang referenziert werden.

\subsection{Dokument}

Ein Dokument hält die Daten und Metadaten einer Datei vor,
beispielsweise einer PDF-Datei, eines RTF- oder Word-Dokuments. Wird von
einem Word-Dokument eine PDF-Ableitung hinterlegt, ist diese Ableitung
ebenfalls ein Dokument, das jedoch nicht als Master gekennzeichnet wird,
sondern auf den entsprechenden Master verweist.

\subsection{Ort}

Dieser Objekttyp dient dazu, einen Ortsbezug einer Drucksache formal
abzubilden. Ortsangaben können sowohl aus Textinformationen bestehen
(beispielsweise der Name einer Straße/eines Platzes oder eine genaue
Adresse) oder aus einer Geo-Koordinatenangabe aus Längen- und
Breitengrad.

\section{Glossar}

\begin{description}
\item[AGS]
Amtlicher Gemeindeschlüssel
\item[RIS]
Ratsinformationssystem
\end{description}

\section{Fußnoten}

{[}1{]}: Siehe
\href{https://www.destatis.de/DE/Methoden/Klassifikationen/Bevoelkerung/StaatsangehoerigkeitGebietsschluessel.html}{www.destatis.de/\ldots{}}
