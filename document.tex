Lizenz: Creative Commons CC-BY-SA

\section{Einleitung}

\subsection{Ziel dieses Dokuments}

Ziel dieses Dokuments ist es, einen Diskurs über einen offenen Standard
zum Datenabruf aus Ratsinformationssystemen in Gang zu bringen.

Ratsinformationssysteme (RIS) werden von vielen Körperschaften wie
Kommunen, Landkreisen und Regierungsbezirken eingesetzt, um die
anfallende Gremienarbeit (Ratssitzungen, Ausschüsse, Vertretungen) zu
organisieren. Da ein großer Teil der schriftlichen Arbeit der
Lokalpolitik über derartige Systeme verwaltet wird, sind die RIS -- dort
wo vorhanden -- ein wichtiger Zugriffspunkt für alle, die sich für
politischen Geschehnisse interessieren.

Eine wichtige Maßnahme von Körperschaften, die im Zuge von Open-Data-
und Open-Government-Initiativen ihre Politik transparenter machen
wollen, wird auch sein, die Daten in den RIS im Sinne des
Open-Data-Begriffs zugänglich zu machen. Hierdurch können die Kommunen
selbst, aber auch dritte, Anwendungen entwickeln, die Inhalte auf
verschiedene Art und Weise auswerten, abrufbar und nutzbar machen, sei
es für die Allgemeinheit oder für bestimmte Nutzerkreise.

In Deutschland gibt es über 10.000 Kommunen, außerdem hunderte weitere
Körperschaften, die über RIS-ähnliche Systeme verfügen. Sollten diese
beginnen, ihre RIS zu öffnen, werden sie sämtlich vor der Frage stehen,
wie Daten zu modellieren und Schnittstellen (APIs) zu spezifizieren
sind.

Sowohl die Anbieter der Daten, als auch die Abnehmer (die
Anwendungsentwickler) werden von einer Standardisierung der
Schnittstellen und Datenmodelle profitieren. So wird die Kompatibilität
von Software und die breite Einsetzbarkeit ermöglicht.

Dieses Dokument soll die Erarbeitung eines solchen Standards ermöglichen
und als Diskussionsgrundlage dienen.

\subsection{Status}

Dieser Entwurf gibt aktuell einen Vorschlag des Autors wieder. Bisher
ist noch kein Feedback eingeflossen.

\subsection{Überblick}

Der Entwurf umfasst im ersten Schritt die abstrakte Beschreibung eines
Datenmodells.

\subsection{Nächste Schritte}

\begin{itemize}
\item
  Bis Ende Januar 2012: Einsammeln von Feedback zum Entwurf des
  Datenmodells Anpassen des Entwurfs anhand von Feedback
\item
  Erarbeitung eines Entwurfs für eine REST-Schnittstelle
\end{itemize}

\subsection{Adresse für Feedback}

Feedback kann gerne per Mail an marian@sendung.de übermittelt werden.
Wer Feedback übersendet, wird als Mitwirkender in zukünftigen Versionen
des Dokuments namentlich erwähnt. Wer dies nicht möchte, sollte dies
bitte in seiner Mail bitte explizit erwähnen.

\section{Datenmodell}

Das Datenmodell soll die Bausteine für die später zu entwerfende
Schnittstelle definieren. Im folgenden werden sozusagen die Objekttypen
bzw. die Klassen beschrieben, auf die über eine spätere API zugegriffen
werden kann.

Einige Objekte werden eine eindeutige Identifizierung (ID) benötigen,
wobei „eindeutig`` auch eine Frage des Kontextes ist. In den wenigsten
Fällen wird es notwendig sein, eine Objekt-Kennung weltweit eindeutig zu
machen. Darüber hinaus wird zu entscheiden sein, ob IDs unveränderlich
oder veränderlich sein sollen.

Die Hinweise auf die Praxis in bestehenden Ratsinformationssystemen
beziehen sich auf nach außen, bei Nutzung der Weboverfläche,
feststellbare Eigenschaften. Es wird auf die folgenden Systeme Bezug
genommen:

\begin{itemize}
\item
  Stadt Köln {[}2{]} - Plattform: Somacos SessionNet {[}3{]}
\item
  Bezirksverwaltung Berlin Mitte {[}4{]} - Plattform: ALLRIS {[}5{]}
\item
  Stadt Rösrath {[}6{]} - Plattform der Firma PROVOX {[}7{]}
\item
  Stadt Euskirchen {[}8{]} - Plattform: SD.NET RIM 4 {[}9{]}
\end{itemize}

Eigenschaften der einzelnen Objekttypen sind, wenn nicht anders
angegeben, verpflichtend. Optionale Eigenschaften sind entsprechend
gekennzeichnet.

Bei Beschreibung der Beziehungen zwischen Objekten wird zu diesem
Zeitpunkt nicht berücksichtigt, ob eine Beziehung zwischen zwei Objekten
A und B am Objekt A oder am Objekt B definiert wird. So spielt es
bislang keine Rolle, ob einem Gremium mehrere Personen zugeordnet werden
oder einer Person mehrere Gremien zugewiesen werden. Das Augenmerkt
liegt hier nur auf der Tatsache, welche Beziehung existieren können und
was diese Beziehungen aussagen sollen.

\subsection{Gebietskörperschaft}

Die Gebietskörperschaft erlaubt es, Körperschaften wie einen bestimmten
Landkreis, eine bestimmte Gemeinde oder einen bestimmten Stadtbezirk in
Form eines Datenobjekts abzubilden.

Viele RIS werden nur genau eine Instanz dieses Typs „beherbergen``.
Denkbar ist aber auch, dass Systeme für einen Verbund von mehreren
Körperschaften betrieben werden.

\begin{figure}[htbp]
\centering
\includegraphics{images/01.png}
\caption{Abbildung: Gebietskörperschaft}
\end{figure}

\subsubsection{Eindeutige Identifizierung}

Zur Identifizierung des Objekts kann der Amtliche Gemeindeschlüssel
(AGS{[}1{]}) verwendet werden, der für Deutschland alle Gemeinden,
Landkreise, kreisfreien Städte etc. eindeutig erfasst.

Vorteil der Verwendung des AGS:

\begin{itemize}
\item
  Kompakte, einfache und einheitliche Schreibweise für jede
  Körperschaft.
\item
  Der AGS wird von Behörden genutzt, ist anerkannt und auch in anderen
  Medien, z.B. der Wikipedia, verbreitet.
\end{itemize}

Nachteil des AGS:

\begin{itemize}
\item
  Führende Nullen machen den Schlüssel fehleranfällig. Bestimmte Systeme
  wie z.B. Excel könnten den Inhalt als Zahlenwert erkennen und die
  führenden Nullen automatisch verwerfen.
\end{itemize}

\subsubsection{Eigenschaften}

\begin{description}
\item[Name]
Der Name der Gebietskörperschaft, z.B. ``Köln'' oder ``Stadt Köln''.
\end{description}

\subsubsection{Beziehungen}

\begin{itemize}
\item
  Objekte vom Typ ``Organisation'' sind zwingend genau einer
  Gebietskörperschaft zugeordnet. So wird beispielseise eine SPD in Köln
  von einer SPD in Leverkusen unterschieden.
\item
  Objekte vom Typ ``Gremium'' sind zwingend einer genau einer
  Gebietskörperschaft zugeordnet. Damit wird der ``Rat'' einer
  bestimmten Kommune von den gleichnamigen Gremien anderer Kommunen
  abgegrenzt.
\end{itemize}

\subsection{Gremium}

Das Gremium ist ein Personenkreis, üblicherweise von gewählten und/oder
ernannten Mitgliedern. Beispiele hierfür sind der Stadtrat, Kreisrat,
Gemeinderat, Ausschüsse und Bezirksvertretungen. Gremien halten
Sitzungen ab, zu denen die Gremien-Mitglieder eingeladen werden.

\subsubsection{Eigenschaften}

\begin{description}
\item[Kennung]
Zur eindeutigen Identifizierung des Gremiums im Kontext einer bestimmten
Gebietskörperschaft. Die Stadt Köln verwendet beispielswiese das Kürzel
``STA'' für den Stadtentwicklungsausschuss oder ``BA'' für den Ausschuss
für Anregungen und Beschwerden. Andere Kommunen verwenden z.B. rein
numerische Kennungen.
\item[Name]
Der Name des Gremiums. Beispiele: ``Rat'', ``Hauptausschuss'',
``Bezirksvertretung 1 (Innenstadt)''
\end{description}

\paragraph{Anmerkungen}

Beim Rösrather RIS {[}6{]} wird für jedes Gremium ein Kurz- und ein
Langname angegeben. Beispielsweise wird beim ``Stadtentwicklungs-,
Planungs- und Verkehrsausschuss'' die kurze Form ``Stadtentwicklung''
hinterlegt. Bei 5 von 12 Gremien sind jedoch Kurz- und Langnamen
identisch.

Sofern nicht Beispiele aus weiteren Systemen vorliegen, wird dieser
Einzelfall nicht im Entwurf abgebildet.

\subsubsection{Beziehungen}

\begin{itemize}
\item
  Objekte vom Typ ``Person'' referenzieren auf Gremien, um die
  Mitgliedschaft/Zugehörigkeit einer Person im/zum Gremium zu
  kennzeichnen.
\item
  Objekte vom Typ ``Drucksache'' können einem Gremium zugeordnet sein.
  Beispielsweise wird eine Anfrage oder ein Antrag dem Rat oder einer
  bestimmten Bezirksvertretung zugeordnet.
\end{itemize}

\subsection{Person}

Jede natürliche Person, die Mitglied eines Gremiums ist, ist als Person
im Datenmodell eindeutig identifizierbar.

\subsubsection{Eigenschaften}

\begin{description}
\item[Kennung]
Zur eindeutigen Identifizierung sollte jede Person eine Kennung
besitzen, die keinen Änderungen unterworfen ist und aus diesem Grund
nicht mit dem Namen in Verbindung stehen sollte. Viele RIS nutzen rein
numerische Kennungen.
\item[Vorname]
Der Vorname der Person.
\item[Nachname]
Der Nachname der Person.
\item[Titel]
\emph{Optional}. Akademische Titel wie ``Dr.'' und ``Prof.~Dr.''
\item[Geschlecht]
\emph{Optional}. Männlich/Weblich
\item[Berufsbezeichnung]
\emph{Optional}. Z.B. ``Rechtsanwalt''
\item[Partei]
\emph{Optional}. Z.B. ``Bündnis 90/Grüne''
\item[E-Mail-Adresse]
\emph{Optional}.
\item[Telefon]
\emph{Optional}.
\item[Fax]
\emph{Optional}.
\item[Anschrift]
\emph{Optional}. Straße und Hausnummer, Postleitzahl und Ort
\end{description}

\paragraph{Anmerkungen}

\begin{itemize}
\item
  Das System von Euskirchen scheint Vor- und Nachname (evtl. einschl.
  Titel) in einem gemeinsamen Feld ``Name'' zu führen. Ob das System
  hier technisch differenziert, ist unklar. Falls einzelne Systeme den
  angezeigten Namen nur als ganzes Speichern, sollte dies für den
  Standard übernommen werden, da es für die meisten Anwendungen
  ausreichen sollte.
\item
  Das Rösrather System kennzeichnet, ob Anschriften privat oder
  geschäftlich sind.
\end{itemize}

\subsubsection{Beziehungen}

\begin{itemize}
\item
  Objekte vom Typ ``Person'' können einer Organisation, z.B. einer
  Fraktion, zugeornet werden. Diese Beziehung ist datiert.
\item
  Objekte vom Typ ``Person'' können einem oder mehreren Gremien
  zugewiesen werden, um die Mitgliedschaft in diesem Gremium
  darzustellen. Diese Beziehungen sind ebenfalls datiert.
\end{itemize}

\subsection{Organisation}

Organisationen sind üblicherweise Parteien bzw. Fraktionen, denen die
Personen angehören können.

\subsubsection{Eigenschaften}

\subsubsection{Beziehungen}

\subsection{Sitzung}

Eine Sitzung ist die Versammlung der Mitglieder eines Gremiums zu einem
bestimmten Zeitpunkt. Sitzungen können eine laufende Nummer haben.,
üblicherweise beginnend bei 1 zu Beginn einer Wahlperiode, haben.

Die geladenen Teilnehmer der Sitzung sind jeweils als „Person`` in
entsprechender Form referenziert. Verschiedene Drucksachen (Einladung,
Ergebnis- und Wortprotokoll) werden ebenfalls referenziert.

\subsubsection{Eigenschaften}

\subsubsection{Beziehungen}

\subsection{Tagesordnungspunkt}

Der Tagesordnungspunkt wird für eine bestimmte Sitzung angelegt, erhält
eine (innerhalb dieser Sitzung eindeutige) Nummer und einen Titel
(Betreff). Nach der Sitzung wird dem Tagesordnungspunkt außerdem ein
Ergebnis angehängt. Falls abweichend von der ursprünglichen
Beschlussvorlage (z.B. durch Berücksichtigung eines Änderungsantrags)
kann ein bestimmter Beschlusstext zu Protokoll gegeben werden. Sofern
das Abstimmungsergebnis nicht einstimmig ist, kann es durch mehrere
referenzierende Stimmabgaben festgehalten werden.

\subsubsection{Eigenschaften}

\subsubsection{Beziehungen}

\subsection{Stimmabgabe}

Wie eine Person bzw. eine Fraktion zu einem Tagesordnungspunkt
abgestimmt hat, wird durch eine Stimmabgabe festgehalten. Ganze
Abstimmungsergebnisse bestehen überlicherweise aus mehreren
Stimmabgaben. Jede Stimmabgabe gibt entweder die (einzelne) Stimme einer
Peson wieder, in diesem Fall ist die Anzahl der Stimmen zwingend 1. Oder
eine Stimmabgabe gibt das Abstimmungsverhalten einer ganzen Gruppe von
Personen wieder. Dann ist die Anzahl der Stimmen anzugeben und statt
einer Person eine Organisation (in der Regel die Fraktion) zu
referenzieren.

\subsubsection{Eigenschaften}

\subsubsection{Beziehungen}

\subsection{Drucksache}

Eine Drucksache bildet Mitteilungen, Antworten auf Anfragen,
Beschlussvorlagen, Anfragen und Anträge ab. Jede Drucksache erhält eine
eindeutige Kennung. Das Datum gibt an, wann die Drucksache erzeugt bzw.
veröffentlicht wurde.

Die Drucksache verweist auf genau ein Hauptdokument. Darüber hinaus
können beliebig viele Dokumente als Anhang referenziert werden.

\subsubsection{Eigenschaften}

\subsubsection{Beziehungen}

\subsection{Dokument}

Ein Dokument hält die Daten und Metadaten einer Datei vor,
beispielsweise einer PDF-Datei, eines RTF- oder Word-Dokuments. Wird von
einem Word-Dokument eine PDF-Ableitung hinterlegt, ist diese Ableitung
ebenfalls ein Dokument, das jedoch nicht als Master gekennzeichnet wird,
sondern auf den entsprechenden Master verweist.

\subsubsection{Eigenschaften}

\subsubsection{Beziehungen}

\subsection{Ort}

Dieser Objekttyp dient dazu, einen Ortsbezug einer Drucksache formal
abzubilden. Ortsangaben können sowohl aus Textinformationen bestehen
(beispielsweise der Name einer Straße/eines Platzes oder eine genaue
Adresse) oder aus einer Geo-Koordinatenangabe aus Längen- und
Breitengrad.

\subsection{Noch nicht abgedeckt}

\begin{itemize}
\item
  Angaben von Personen zu Tätigkeiten (z.B. Auskunft nach § 17
  Korruptionsbekämpfungsgesetz)
\end{itemize}

\section{Glossar}

\begin{description}
\item[AGS]
Amtlicher Gemeindeschlüssel
\item[RIS]
Ratsinformationssystem
\end{description}

\section{Fußnoten}

{[}1{]}: Siehe
\href{https://www.destatis.de/DE/Methoden/Klassifikationen/Bevoelkerung/StaatsangehoerigkeitGebietsschluessel.html}{www.destatis.de/\ldots{}}

{[}2{]}: Ratsinformationssystem der Stadt Köln,
\href{http://ratsinformation.stadt-koeln.de/}{http://ratsinformation.stadt-koeln.de/}

{[}3{]}: Firma Somacos,
\href{http://www.somacos.de/de/sitzungsdienst/ratsinfo.html}{SessionNet
Produktinformation}

{[}4{]}: Ratsinformationssystem der Bezirksverwaltugn Berlin Mitte,
\href{http://www.berlin.de/ba-mitte/bvv-online/allris.net.asp}{http://www.berlin.de/ba-mitte/\ldots{}}

{[}5{]}: CC e-gov GmbH, \href{http://www.cc-egov.de/allris.htm}{ALLRIS
Produktionformationen}

{[}6{]}: Ratsinformationssystem der Stadt Rösrath,
\href{http://212.227.97.55/ratsinfo/roesrath}{http://212.227.97.55/ratsinfo/\ldots{}}

{[}7{]}: \href{http://www.provox.de/}{Firma PROVOX}

{[}8{]}: Ratsinformationssystem der Stadt Euskirchen,
\href{https://sitzungsdienst.euskirchen.de/}{https://sitzungsdienst.euskirchen.de/}

{[}9{]}: Firma Sternberg,
\href{http://www.sitzungsdienst.net/produkte/ratsinformationsmanagement}{SD.NET
RIM Produktionformationen}
